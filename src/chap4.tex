\section{重修线性代数4——表达} 
抽象的线性空间涵盖许多不同的数学实体,而它们可以表达成列向量和矩阵来计算。将抽象空间中的运算一一对应到一个具体空间中来研究,这是学习抽象方法的一个重要入口。习惯了它,在学习抽象代数时,你就能自然地欣赏同构,同态和商空间运用的美妙。

\subsection{基与坐标}

线性代数课程主要在讲矩阵的运算。这是为什么?列向量和矩阵仅仅是抽象线性空间中向量和线性算子的一个具体实现,满足抽象定义的具体数学实体有无数种,线性空间和算子的许多证明都直接在矩阵上讲,而不是根据一般的抽象定义来证明,这样的结论会具有普遍性,能适用于其他吗?

这里的原因,一是作为抽象数学的入门课,线性代数课从列向量和矩阵开始的学习内容,提供了一个从直观的构造主义传统数学,到抽象的形式主义公理化现代数学思想方法的过渡。另一是,列向量和矩阵可以看成各种向量和线性算子的表达,它们之间建立起的一一对应关系,在线性运算中也保持,在线性代数的结构上是等价的,所以相关结论也适用于其他。

如果一个线性空间X里线性无关向量组里最多有n个向量,n是有限的,那么它称为\textbf{n维线性空间}。记为$ dim(X)=n $.  例如长度为n数组的线性空间是n维的。最高阶小于n的多项式空间是n维的。

两个数学空间X,Y,如果它们间的元素存在着一个一一满映射σ,对抽象空间中的运算φ有 % MathType!MTEF!2!1!+-
% feaahqart1ev3aaatCvAUfeBSjuyZL2yd9gzLbvyNv2CaerbuLwBLn
% hiov2DGi1BTfMBaeXatLxBI9gBaerbd9wDYLwzYbItLDharqqtubsr
% 4rNCHbWexLMBbXgBd9gzLbvyNv2CaeHbl7mZLdGeaGqiVu0Je9sqqr
% pepC0xbbL8F4rqqrFfpeea0xe9Lq-Jc9vqaqpepm0xbba9pwe9Q8fs
% 0-yqaqpepae9pg0FirpepeKkFr0xfr-xfr-xb9adbaqaaeGaciGaai
% aabeqaamaabaabauaakeaacaqGGaGaeqy1dyMaaiikaiabeo8aZjaa
% dIhacaGGSaGaeq4WdmNaamyEaiaacMcaaaa!4913!
\[{\rm{ }}\phi (\sigma x,\sigma y)\] ,即其一空间中的运算完全对应于在另一个空间中的运算,则称为它们是\textbf{同构}的。如果忽略同构对象的属性和运算的具体定义,就结构而言,同构的对象是等价的。相同维数线性空间之间的满秩线性算子,即是一个能够保持线性运算的一一满映射,所以相同维数的线性空间都是同构的。

同构意味空间的数学结构是一样的,在抽象代数学习中,可以用其中一个简单熟悉的数学实体作为代表来作研究。对于线性空间,列向量和矩阵就是这个的代表,有限维线性空间的计算和性质可以通过它们的计算来实现。

向量和线性算子都能表示成数域上的列向量和矩阵。这列向量在有限维的线性空间是有限的数组,在无穷维的空间则是无穷级数中的系数或积分中的加权函数项。下面介绍其映射关系。

取n维线性空间X中一组n个线性无关的向量,$ {e_1,e_2,\cdots,e_n} $,称为一组基,按最大线性无关组的定义,X中任何向量都可以分解为基向量的线性组合,

\begin{gather*}
	{r} = r_1{e_1} + r_2{e_2}+ \cdots + r_n{e_n} =\begin{pmatrix} {e_1} &{e_2}  &\dots  & {e_n}\end{pmatrix} \begin{pmatrix}r_1 \\ r_2 \\ \vdots \\r_n \end{pmatrix}
\end{gather*}

向量$ r $对这组基分解的系数,称为向量在这组基下的坐标,它们是一组数,把它排成一列按矩阵的形式运算,表示为上面的列向量$ (r_i)_n $. 列向量线性空间$ \mathbb{R}^n $的基,是$ n $个只有一个分量为1,其余都为0的数组集合。把列向量看成坐标的数组,在列向量作为线性空间X中向量坐标表示的映射中,$ X $的基向量$ ei $对应着$ \mathbb{R}^n $的,只有第i分量是1的基向量。

给定一组基下的线性表示,n维线性空间X中的向量与数域上n阶的列向量建立起一个一一对应的满映射。不难证明这种对应关系,对向量的加法和数乘也保持。相同数域上的相同维数的线性空间都是对线性运算同构的。这意味着n维线性空间中的线性运算,都可以通过映射对应到对列向量的运算。

显然在不同的基上,同一个向量的坐标表示是不同的。

\kaishu

例如所有的小于5次的多项式构成一个5维线性空间,$ 1, x, x^2, x^3,  x^4 $, 是一组基,所有小于5次的多项式都可以表示成它们的线性组合。式子$ x+1, x-1, (x-1)^2,(x-1)^3,(x-1)^4 $也是一组基,这空间里的多项式也都可以表示为它们的线性组合。

\songti

设$ \alpha $是n维线性空间X到m维线性空间Y的一个线性算子,$ {e_1,e_2,\cdots,e_n} $ 和
  $ {g_1,g_2,\cdots,g_m} $ 分别是X和Y上的一组基,$ \alpha e_j $是Y上的一个向量,可以表示为 $ a_{1_{j}}g_1+a_{2_{j}}g_2,+\cdots+a_{m_j}g_m $,所以对X上的任意向量r,我们有

\begin{gather*}
	\alpha\textbf{r}=\alpha \sum_{j=1}^n r_j \textbf{e}_j = \sum_{j=1}^n r_j \alpha \textbf{e}_j = \sum_{j=1}^n r_j \sum_{i=1}^m a_{ij}\textbf{g}_i = \sum_{i=1}^m( \sum_{j=1}^n a_{ij} r_j)\textbf{g}_i	
\end{gather*}

将等式右边表示成矩阵形式就是:

\begin{gather*}
	\begin{pmatrix}\textbf{g}_1,& \textbf{g}_2,& \cdots,& \textbf{g}_m \end{pmatrix}\begin{pmatrix} a_{1,1}& a_{1,2}&\cdots& a_{1,n} \\ a_{2,1}& a_{2,2}&\cdots& a_{2,n} \\ \vdots& \vdots & \ &\vdots \\ \vdots& \vdots & \ &\vdots  \\a_{m,1}& a_{m,2} &\cdots& a_{m,n} \end{pmatrix} \begin{pmatrix}r_1 \\r_2 \\ \vdots \\r_n \end{pmatrix}
\end{gather*}

n维到m维线性空间上的线性算子$ \alpha $,在给定的基下表示为一个m*n数域上的矩阵A,矩阵中n个列向量,分别对应于$ \alpha $将n维空间每个基向量,映射到m维空间上向量的坐标表示。这个线性算子对向量r的作用所得向量的坐标列向量,等于矩阵A乘以r的坐标列向量。类似的,可以证明复合线性算子的矩阵表示,等于它们表示矩阵相乘所得的矩阵。

同一个线性空间上的线性算子对加法和数乘构成一个线性空间,它与算子的矩阵表示构成的线性空间,对线性运算和乘法运算都是同构的。所以通过基形成的对应关系,线性空间的向量和线性算子的运算都可以通过列向量和矩阵的运算来表示。

\kaishu

函数求导是一个线性运算,它也是在那个小于5次的多项式线性空间上的一个线性变换,在 ,$ 1, x, x^2, x^3,  x^4 $,的基下,它可以表示为一个矩阵D,多项式$ x^4+2x^3-3x^2+x+12 $在这组基下表示为向量r,对这多项式求导得到$ 4x^3+6x^2-6x+1 $,也可以从下面矩阵运算中体现。

\begin{gather*}
	D=\begin{pmatrix}0&0&0&0&0\\4&0&0&0&0\\0&3&0&0&0\\0&0&2&0&0\\0&0&0&1&0 \end{pmatrix},r=\begin{pmatrix}1\\2\\-3\\1\\12\end{pmatrix},  Dr=\begin{pmatrix}0\\4\\6\\-6\\1\end{pmatrix}
\end{gather*}

请在$ x+1, x-1, (x-1)^2,(x-1)^3,(x-1)^4 $这组基下写出这例子中的矩阵、向量并验算求导的答案。问这矩阵的秩是多少,给出一个零空间中的向量。

\songti

在你头脑的图像中,向量、基向量、线性算子所指的,是有向线段、坐标轴、多项式、函数等等数学世界符合抽象定义规范的东西。而这些向量和线性算子,对应着在基坐标的列向量和矩阵,是它们在数值世界上的影像。在不需要区分时,人们经常直接用列向量和矩阵代表所对应的向量和线性算子。在不同的基下,同一个向量和线性算子有不同的列向量和矩阵表示,这就像我们用不同的角度来描述同一个物体一样。它们之间对应着一个坐标变换。

\subsection{坐标变换}

满映射的线性变换,称为是满秩的。满秩线性变换矩阵中的列向量组是线性无关的,其矩阵的行列式不等于零。当表达线性空间X从一组基 $ {e_1,e_2,\cdots,e_n} $ 换成另外一组新的基 $ {g_1,g_2,\cdots,g_n} $ 时,同一个向量r对应于这旧的基上的坐标列向量 $ (r_i)_n $ 也相应变化成新的坐标列向量 $ (h_i)_n $ . 

记n维线性空间X中两组基间的线性变换为T,它把旧的基$ {e_1,e_2,\cdots,e_n} $变换成新的基$ {g_1,g_2,\cdots,g_n} g_i = \mathbb{T}e_i, i=1,2,\cdots,n$ ,这可以表示成一个满秩的n阶方阵$ \mathbb{T} $,其中的列向量对应着新的基向量在旧的基下的坐标。

\begin{gather*}
	\begin{pmatrix}{g}_1& {g}_2& \cdots & {g}_n \end{pmatrix} = \begin{pmatrix} \mathbb{T}\textbf{e}_1& \mathbb{T}{e}_2& \cdots & \mathbb{T}{e}_n \end{pmatrix} =\begin{pmatrix} {e_1} &{e_2}  &\dots  & {e_n}\end{pmatrix}T
\end{gather*}

它将X中向量r在新的基下坐标的变换成旧的基下的坐标。

\begin{gather*}
	{r}= \begin{pmatrix} {e_1} &{e_2}  &\dots  & {e_n}\end{pmatrix} \begin{pmatrix}r_1 \\ r_2 \\ \vdots \\r_n \end{pmatrix} = \begin{pmatrix}{g}_1 & \textbf{g}_2& \cdots &\textbf{g}_m \end{pmatrix}\begin{pmatrix}h_1 \\ h_2 \\ \vdots \\h_n \end{pmatrix}=\\
	\begin{pmatrix} {e_1} &{e_2}  &\dots  & {e_n}\end{pmatrix} T \begin{pmatrix} h_1 \\ h_2 \\ \vdots \\h_n \end{pmatrix} \Rightarrow \begin{pmatrix}r_1 \\ r_2 \\ \vdots \\r_n \end{pmatrix} = T\begin{pmatrix}h_1 \\ h_2 \\ \vdots \\h_n \end{pmatrix}
\end{gather*}

记 $ T_i $ 是新的基向量 $ g_i $ 在旧的基 $ {e_1,e_2,\cdots,e_n} $ 上坐标的列向量,矩阵 $ T=(T_1, T_2,\cdots,T_n) $ ,它是由新的基向量在旧的基地上坐标列向量排成一行的矩阵。向量在旧的基上坐标的列向量,是新的基上坐标的列向量,对新旧基表示的列向量$ {T_1, T_2,\cdots,T_n} $的线性组合系数。

不难看出对应于不同基坐标的列向量之间是个满秩的线性变换。$ T^{-1} = (F_1,F_2, \cdots, F_n) $,其中 $ (F_i)_n $ 是旧的基向量 $ e_i $ 在新的基上坐标的列向量。满秩的方阵也代表着一个坐标变换。


\subsection{坐标变换中的矩阵表示}

线性算子是从一个线性空间到另一个线性空间的映射。算子$ \alpha $的作用将$  $n维线性空间$ X $中的向量,变成$ m $维线性空间$ Y $中的向量,在给定这两个空间的基上,它表示为一个$ m*n $矩阵$ A $。两个线性空间的基是各自独立的,可以各自分别更换。继续沿着向量坐标变换的思路可以看出,在两个空间基的更换中,如果$ X $中坐标变换的矩阵是$ V $,在$ Y $中是$ U $,在新的两个基上,$ \alpha $的矩阵表示是$ U^{-1}AV $.

通过满秩变换P和Q,把一个矩阵A变为另一个矩阵B=PAQ,称矩阵A和B是\textbf{等价}的。P和Q都是满秩的方阵,$ P^{-1} $和Q分别可以看成是算子输入和输出空间的坐标变换,等阶的矩阵可以看成是同一个线性算子在不同坐标系下的表示。

线性变换是线性空间到自身的映射。变换的输入和输出在同一个空间。在给定的基上,它的作用表示为一个方阵A。基的更换不仅影响着变换作用输入的向量表示,同样影响了输出的向量表示,所以当新旧基间坐标变换为T,在新的基上,矩阵的表示将是$ T^{-1}AT $,它与旧基上的A是相似的矩阵。

任何满秩的方阵T都对应着一个坐标变换,矩阵$ B=T^{-1}AT $,称A和B是\textbf{相似}的,说明它们可以通过坐标变换来相等。相似的矩阵是同一个线性变换在不同基上的表示。

对于一个方阵A,它是既可以表示一个线性变换,也可以是一个线性算子,当它代表一个线性算子时,输入X和输出Y是相同维数的不同空间。它们的坐标变换是分别独立的。

\subsection{标准正交基}

向量对一组基的线性组合分解是唯一的,这直接可以从基的线性无关性来证明。但这只是说明这个线性空间与相同维数列向量的欧几里德空间是一一对应,而且是同构的。这并不意味着这种映射是唯一的,即不意味着对应的坐标是确定的。到目前为止,我们都不确定向量在基上的坐标,因为这涉及到向量与基向量之间的关系。引入两个向量间关系的内积概念,将让这个坐标映射清晰起来。

假设向量表达成基向量的线性组合的式子,用基向量对这表达式两边取内积,得到一个以坐标为未知数的线性方程组,它的解便确定了向量在这基上的坐标。

如果基中的向量长度都为1,并且相互正交,它则称为\textbf{标准正交基}。向量在标准正交基上的坐标等于向量在那个基向量上的投影,即它与基向量的内积。如果两组基都是标准正交基,那么它们间的变换T便是一个正交阵(酉阵对于复数域),它的转置(共轭转置对于复数域)便是它的逆,把向量新坐标变成它旧的坐标。

列向量只有在作为标准正交基坐标上的向量表示时,它的内积才等于所表示向量的内积。在一般情况下,基向量的内积构成一个正定矩阵 $ E = ( e_1, e_2, \cdots$  $, e_n )$$  T $ $ \cdot(e_1,e_2,\cdots,e_n) $ ,
向量$ x, y $的内积等于它们在这基上坐标表示列向量 $ (x_i)$ ,  $(y_i) $ 与它的乘积 $(\bar{x}_i)^TE(y_i) $ . 所以,列向量作为内积空间向量的坐标表示时,它是在标准正交基上的表示。用标准正交基上的向量坐标表示,内积等于一个列向量共轭转置与另一个列向量的乘积。这样列向量所在的坐标空间与内积空间才是同构的。

前面说,满秩的线性算子建立起相同维数的线性空间的同构关系,这只是对线性运算而言。但这并不足以保证对应的内积运算也相等。内积空间还包含着内积的运算,能保持内积空间同构的线性算子,是正交阵(酉阵对于复数域)。从几何直观来看,在线性算子作用下保持向量的长度和夹角不变(即内积不变),这个作用只能是旋转和翻转的组合。

这一小节陈述了着线性空间坐标变换关系的直观图像。如果读者不能在脑海中清晰地看到,建议花点时间,在代数表达式推演的帮助下来消化理解。有了对同一个向量和线性变换,在不同基下列向量和矩阵坐标变换的直观理解,将会比较容易看清抽象的概念与具体的数值表示间的关系。